\documentclass{article}
\usepackage{graphicx} % Required to include images
\usepackage{caption}  % For captions and table numbering
\usepackage{array}    % For more advanced table options

\begin{document}

\noindent d. Graph the graph below, label the y --- axis, x --- axis, y --- intercept, and x --- intercept:
\vspace{1em} % Add some vertical space

\begin{figure}[htbp]
    \begin{minipage}[t]{0.6\textwidth}
        \centering
        % Replace 'your_graph_image.png' with the actual filename of your graph image.
        % Make sure the image is in the same directory as your .tex file.
        % \includegraphics[width=\textwidth]{your_graph_image.png}
        \caption*{A blank graph for plotting} % Use a caption if you want to label the graph.
    \end{minipage}
    \begin{minipage}[t]{0.4\textwidth}
        \centering
        \begin{tabular}{|c|c|}
            \hline
            $t$ & $L(t)$ \\
            \hline
            0 & \\
            \hline
            3 & \\
            \hline
            5 & \\
            \hline
            11 & \\
            \hline
            15 & \\
            \hline
            18 & \\
            \hline
            20 & \\
            \hline
        \end{tabular}
    \end{minipage}
\end{figure}

\vspace{2em} % Add more vertical space

\noindent 2. You are in a race, you have a 6 foot headstart and are running a short 100 ft race
\vspace{1em}

% "Sideways" or horizontal table using the tabular environment
\begin{center}
    \begin{tabular}{|c|c|c|c|c|c|c|c|}
        \hline
        Seconds & 0 & 1 & 2 & 3 & 4 & 5 & 6 \\
        \hline
        Feet & 6 & 8 & 10 & 12 & 14 & 16 & 18 \\
        \hline
    \end{tabular}
\end{center}

\noindent a. Use any two of the ordered pairs from the table to derive the equation $F$ in terms of $t$

\end{document}