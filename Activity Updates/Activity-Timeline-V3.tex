\documentclass[12pt]{article}
\usepackage{amsmath}
\usepackage{amssymb}
\usepackage{fancyhdr}
\usepackage{enumitem}

\usepackage[
  letterpaper,
  left=0.8in,
  right=0.8in,
  textheight=9.5in,
  bmargin=0.5in  % Adjust this value to push the footer down
]{geometry}
\pagestyle{fancy}
\fancyhf{} % Clear all header and footer fields
\fancyhead[L]{Your Name:} % Left header with name
\fancyhead[R]{Novemeber 20th 2025} % Right header with date

\newenvironment{myLectureNotes}{
    % Start the description environment with your specific settings
    \begin{description}[
    font=\normalfont,      % % Ensures the content text is *not* bold (only the label is)
    style=nextline,        % % Forces the content to start on a new line below the label
    leftmargin=25pt,       % % Sets the indentation of the entire content block from the page's left margin
    itemindent=10pt,       % % Sets an additional indent for the content within the item (adds to leftmargin)
    labelsep=0.5em,        % % The horizontal space *between* the label and content (ignored due to style=nextline)
    itemsep=1em,         % % Adds vertical space *between* different \item entries (e.g., between Updates and Topic)
    parsep=0em             % % Controls vertical space *within* an item if the content spans multiple paragraphs
]
}{\end{description}}

\newcommand{\DayNotes}[1]{%
    \section*{#1}
    \begin{myLectureNotes}
}

\newcommand{\EndNotes}{%
    \end{myLectureNotes}
}

\begin{document}

% Main title
\begin{center}
    \Large \textbf{Activity Timeline} \\
    \vspace{0.2cm}
    \normalsize Math-115-02 Activities\\
    \normalsize To keep track of shit because hard
\end{center}



\DayNotes{Day 01 ----------------- (08/19)}
    \item[\quad\textbf{Updates:}] 
    Prof TA LA Faculty Info, Activities built their own Community Guidelines, Name cards,
    Introductions, Slido about Summer, Stem lab visit

    \item[\quad\textbf{Notes:}] 
    Bonnie forgot about activity was a thing until the day before class so nothing was properly integrated or talked about. 
    Syllabus was not completed before the first week, so I/we had to wing some things and assume things on the fly
\EndNotes
\rule[1em]{\linewidth}{0.4pt}\vspace{-1cm}



\DayNotes{Day 02 ----------------- (08/21)} 
    \item[\quad\textbf{Updates:}] 
    Revised Faculity Info, Combined Community Guidlines, Created Discord server, talked about XYZ, add/drop/switching. Made groups for the upcoming weeks. 
    Talked about math bio briefly and learning styles online activity in class and compared them to other's answers
\EndNotes
\rule[1em]{\linewidth}{0.4pt}\vspace{-1cm}



\DayNotes{Day 03 | Act 01 | (08/26)} 
    \item[\quad\textbf{Topic:}] Chapter 1: Section 2: Representing Collections of Numbers

    \item[\quad\textbf{Material:}]
    Interval Notation $(a,b)$ and inequality notation $a\leq x \leq b$ with $\pm \infty$ from a number line

    \item[\quad\textbf{Notes:}] 
    Students had issues remembering solid or empty dot and their respective symbols

\EndNotes
\rule[1em]{\linewidth}{0.4pt}\vspace{-1cm}



\DayNotes{Day 04 ----------------- (08/28)} 
    \item[\quad\textbf{Topic:}] Chapter 2, Section 1: The Coordinate System

    \item[\quad\textbf{Material:}]
    My old classic MATH plotting activity. Got everyone set up on XYZ and explained details

    \item[\quad\textbf{Notes:}] 
    Students were confused of how XYZ works, the difference of options for purchase now vs later, 
    if they can get around the \$45 etc, and how to type math symbols on the keyboard

   \item[\quad\textbf{Updates:}]
   Explained that they get bonus points for going to the Stem Lab!
\EndNotes
\rule[1em]{\linewidth}{0.4pt}\vspace{-1cm}



\DayNotes{Day 05 | Act 02 | (09/02)} 
    \item[\quad\textbf{Topic:}] Chapter 2, Section 1: The Coordinate System
    
    \item[\quad\textbf{Material:}]
    Four simple plots on a graph, find the x-intercepts $(x,0)$ and the y-intercepts $(0,y)$.\\
    Two tables about simple motion, fill in the table blanks and answer the questions about them

    \item[\quad\textbf{Notes:}] 
    They had trouble recognizing some patters $(0,2,4,a,6,8,b,12)$ or $(-13,x,-9,-,7,...,y,1,3,z,7)$ \\
    Making sure they answered in units $"7"$ vs $"7$ feet" wasn't that automatic ever if the question stated find how many are at a certain time.
    The table values representing touching the ground, maximum, intercepts, and a description of a point seemed a bit much for them, they kept mixing up values

   \item[\quad\textbf{Updates:}]
   Syllabus was still not completed and they had too many questions to keep up with 
\EndNotes
\rule[1em]{\linewidth}{0.4pt}\vspace{-1cm}



\DayNotes{Day 06 | Act 03 | (09/04)} 
    \item[\quad\textbf{Topic:}] Chapter 2: Section 2: What are Functions?
    
    \item[\quad\textbf{Material:}]
    Find the domain and range of increasingly harder functions, including piece-wise. \\
    Then given a fixed domain and range, build functions that fit in the regions while \\ 
    still meeting the conditions for an open or closed or peice-wise domain and range

    \item[\quad\textbf{Notes:}] 
    Students confused domain and range now that there were two dimentions, heightest, lowest, most left most right
    got confusing for them to keep track of each part seperately, also knowing when to use the union operation
    Then building their own plots were hard, so many did straight lines, few did diagonal lines that technically worked
    but once it got to piece-wise, many of their plots failed the vertical line test or were just too simple.
    It got tricky when the given domain and range seemed to conflict and keeping track of open or closed.

   \item[\quad\textbf{Updates:}]
   Uhhhh yeah vomit mess in the class before me, had to switch rooms and was so delayed.
   Converted it into a spontaneous homework, and had to explain the rules for the first homework.
   Finally gave them an activity syllabus, before I realized the parent class one was finished without me knowing. 
   It helped but I wish I was more on top of updating canvas
\EndNotes
\rule[1em]{\linewidth}{0.4pt}\vspace{-1cm}



\DayNotes{Day 07 | Act 04 | (09/09)} 
    \item[\quad\textbf{Topic:}] Chapter 3: Section 1: Intro to Function Notation
    
    \item[\quad\textbf{Material:}] 
    Really fun Blackout 5x5 bingo to see who remembers simple algebraic expressions and they find others to sign their name if they know it.
    There were two similar versions played at once so there was more variety and more changes to win bonus points

    \item[\quad\textbf{Notes:}] 
    Many of them forgot so much or were lazy to not even attempt it and only chose the easy ones. 
    Once someone finished after putting their max of two signatures, they were able to 
    leave, which left people at the end of class abandomed and without help. This activity felt 
    necessary since there was no formal way see what math knowledge we were working with

   \item[\quad\textbf{Updates:}]
   Reminded students that their two options for XYZHomework were either to buy online or at the library as their free trials were ending soon while homework was due shortly later
\EndNotes
\rule[1em]{\linewidth}{0.4pt}\vspace{-1cm}



\DayNotes{Day 08 | Act 05 | (09/11)} 
    \item[\quad\textbf{Topic:}] Chapter 3: Section 1: Intro to Function Notation
    
    \item[\quad\textbf{Material:}]
    Introduction to the table of Algebraic Rules and Function Composition Part 1. Many problems divided in 4 levels of difficulty with six problems each. 
    
    
    \item[\quad\textbf{Notes:}] 
    Some were very easy, some where a headache. The students still had trouble with a lot and I half regret making it so hard; they seemed to mostly enjoy it though.
    Thgere were some multiplying by 0 and substitution errors, but I was slowly realizing how much they didnt remember already dispite them having a very useful table they kept not looking at for help.
    A couple of the last problems took forever, but they seemed proud of getting it right or wanting to fix their mistakes 
    
    \item[\quad\textbf{Updates:}]
    I heard about how many students still didnt have XYZ yet and reminded them again trying to be patient, some even forgot what I showed them before
\EndNotes
\rule[1em]{\linewidth}{0.4pt}\vspace{-1cm}



\DayNotes{Day 09 | Act 06 | (09/16)} 
    \item[\quad\textbf{Topic:}] Chapter 3: Section 3: Algebra of Functions
    
    \item[\quad\textbf{Material:}]
    Algebraic Rules Table Revised and Function Composition Part 2. Clarified the binomial expansion formula. 
    Rather than just one function and plugging in another small one, it's now multiple functions in functions which requires more effort
    
    \item[\quad\textbf{Notes:}] 
    The most difficuly activity and concept for them. Would've liked to have switched to the better room at this point to see their thought process.
    They had such a hard time plugging functions into other functions with just their label and value. Some would not expand correctly, 
    some would not do the right order, and others just plain forgot or did not understand

    \item[\quad\textbf{Updates:}]
    Last time I bring up XYZ due to the deadline being that weekend and remind them again how to use it, buy it, and access it. 
    So many still procrastinated and were falling behind
\EndNotes
\rule[1em]{\linewidth}{0.4pt}\vspace{-1cm}


\DayNotes{Day 10 | Act 07 | (09/18)} 
    \item[\quad\textbf{Topic:}] Chapter 3: Section 3: Algebra of Functions
    
    \item[\quad\textbf{Material:}] Evaluating function compositions given three functions given by a table, a function, and a peice-wise graph straight from the guided notes. 
    Then complete the table given a function and use three rows to evaluate more function compositions 
    
    \item[\quad\textbf{Warm Up:}]
    First Warm-Up, either my idea or the professor suggested it. The very infamous 16 different solutions pemdas warm-up then combining like terms. 
    Man oh man we should have gone over basic pemdas and distribution beforehand
    Quite a good addition but later realized how much time it uses for not much pay off. 
    I was able to see their thoughts and errors when they worked alone, that helped guide some of the future activities.
    Unfortunately this added more points to keep track of and did not add much to their end.
    

    \item[\quad\textbf{Notes:}] 
    Students improved a bit, but definitely confused over the different forms of the function and when to use each.
    Hilariously they evaluated a cubic with their pathetic phone calculators once again and/or did each problem before they filled in the table first like they should have

    \item[\quad\textbf{Updates:}]
    Hell yeah no more mentioning XYZ! First few weeks of activity were going to be graded that weekend, took me long enough but I did it over a week or two before the first exam. 
   
\EndNotes
\rule[1em]{\linewidth}{0.4pt}\vspace{-1cm}



\DayNotes{Day 11 | Act 08 | (09/23)} 
    \item[\quad\textbf{Topic:}] Chapter 4: Section 1: Average Rate of Change
    
    \item[\quad\textbf{Material:}]
    This starts the more rigoruous sections of algebra. Given a quadratic, find the average rate of change on a particular interval of its domain.
    Then, using a very nice cubic graph, find the average rats of changes between labeled integer points.
    
    \item[\quad\textbf{Warm Up:}]
    Very simple warm up about adding and multiplying the same two simple fractions with different denominators, oh boy did they mess it up and forget what to do, some got it right
    
    \item[\quad\textbf{Notes:}] 
    Shockingly a lot more difficult than expected. This begins them having to remember definition structure and order. 
    Also, double negative kept being simplified to a negative and more algebratic typos also a fair number struggled still with plugging in numbers into a function.
    
 
\EndNotes
\rule[1em]{\linewidth}{0.4pt}\vspace{-1cm}



\DayNotes{Day 12 | Act 09 | (09/25)} 
    \item[\quad\textbf{Topic:}] Miscellaneous Review
    
    \item[\quad\textbf{Material:}]
    Brought back the Algebraic Rules table and properly went over pemdas, distribution property, 
    multiplying negatives, and squaring negatives. Thought should have been done the first week.
    
    \item[\quad\textbf{Warm Up:}]
    Function evaluation that involved double negatives and muttiplication with positives and negitives, 
    and reminding them that small minus big equals negative. Some of the students showed noticeable improvement

    \item[\quad\textbf{Notes:}] 
    Honestly feels like students forgot how to do calculations by hand and work it out step by step.  
    There were times that they did not even know how to use the table which concerned me more. 
    I understand using the calculator but this was the easiest activity in the whole semester. 
    Of course some flew through it but others lagged significantly behind 
\EndNotes
\rule[1em]{\linewidth}{0.4pt}\vspace{-1cm}



\DayNotes{Day 13 ----------------- (09/30)} 
    \item[\quad\textbf{Topic:}] Hosted Lecture for Exam 1 Review
    
    \item[\quad\textbf{Material:}]
    
    \item[\quad\textbf{Warm Up:}]

    \item[\quad\textbf{Notes:}]
    
    \item[\quad\textbf{Topic:}] Activity Exam 1 Review
    
    \item[\quad\textbf{Material:}]
    
    \item[\quad\textbf{Warm Up:}]

    \item[\quad\textbf{Notes:}] 

    \item[\quad\textbf{Updates:}]
   
\EndNotes
\rule[1em]{\linewidth}{0.4pt}\vspace{-1cm}



\DayNotes{Day 14 ----------------- (10/02)} 
    \item[\quad\textbf{Topic:}] Exam 1, Canceled Class
    
    \item[\quad\textbf{Material:}]
    
    \item[\quad\textbf{Warm Up:}]

    \item[\quad\textbf{Notes:}] 

    \item[\quad\textbf{Updates:}]
   
\EndNotes
\rule[1em]{\linewidth}{0.4pt}\vspace{-1cm}



\DayNotes{Day 12 | Act 09 | (09/25)} 
    \item[\quad\textbf{Topic:}] 
    
    \item[\quad\textbf{Material:}]
    
    \item[\quad\textbf{Warm Up:}]

    \item[\quad\textbf{Notes:}] 

    \item[\quad\textbf{Updates:}]
   
\EndNotes
\rule[1em]{\linewidth}{0.4pt}\vspace{-1cm}



\DayNotes{Day 13 | Act 10 | (09/25)} 
    \item[\quad\textbf{Topic:}] 
    
    \item[\quad\textbf{Material:}]
    
    \item[\quad\textbf{Warm Up:}]

    \item[\quad\textbf{Notes:}] 

    \item[\quad\textbf{Updates:}]
   
\EndNotes
\rule[1em]{\linewidth}{0.4pt}\vspace{-1cm}



\DayNotes{Day 12 | Act 09 | (09/25)} 
    \item[\quad\textbf{Topic:}] 
    
    \item[\quad\textbf{Material:}]
    
    \item[\quad\textbf{Warm Up:}]

    \item[\quad\textbf{Notes:}] 

    \item[\quad\textbf{Updates:}]
   
\EndNotes
\rule[1em]{\linewidth}{0.4pt}\vspace{-1cm}



\DayNotes{Day 12 | Act 09 | (09/25)} 
    \item[\quad\textbf{Topic:}] 
    
    \item[\quad\textbf{Material:}]
    
    \item[\quad\textbf{Warm Up:}]

    \item[\quad\textbf{Notes:}] 

    \item[\quad\textbf{Updates:}]
   
\EndNotes
\rule[1em]{\linewidth}{0.4pt}\vspace{-1cm}



\DayNotes{Day 12 | Act 09 | (09/25)} 
    \item[\quad\textbf{Topic:}] 
    
    \item[\quad\textbf{Material:}]
    
    \item[\quad\textbf{Warm Up:}]

    \item[\quad\textbf{Notes:}] 

    \item[\quad\textbf{Updates:}]
   
\EndNotes
\rule[1em]{\linewidth}{0.4pt}\vspace{-1cm}



\DayNotes{Day 12 | Act 09 | (09/25)} 
    \item[\quad\textbf{Topic:}] 
    
    \item[\quad\textbf{Material:}]
    
    \item[\quad\textbf{Warm Up:}]

    \item[\quad\textbf{Notes:}] 

    \item[\quad\textbf{Updates:}]
   
\EndNotes
\rule[1em]{\linewidth}{0.4pt}\vspace{-1cm}



\DayNotes{Day 12 | Act 09 | (09/25)} 
    \item[\quad\textbf{Topic:}] 
    
    \item[\quad\textbf{Material:}]
    
    \item[\quad\textbf{Warm Up:}]

    \item[\quad\textbf{Notes:}] 

    \item[\quad\textbf{Updates:}]
   
\EndNotes
\rule[1em]{\linewidth}{0.4pt}\vspace{-1cm}



\DayNotes{Day 12 | Act 09 | (09/25)} 
    \item[\quad\textbf{Topic:}] 
    
    \item[\quad\textbf{Material:}]
    
    \item[\quad\textbf{Warm Up:}]

    \item[\quad\textbf{Notes:}] 

    \item[\quad\textbf{Updates:}]
   
\EndNotes
\rule[1em]{\linewidth}{0.4pt}\vspace{-1cm}



\DayNotes{Day 12 | Act 09 | (09/25)} 
    \item[\quad\textbf{Topic:}] 
    
    \item[\quad\textbf{Material:}]
    
    \item[\quad\textbf{Warm Up:}]

    \item[\quad\textbf{Notes:}] 

    \item[\quad\textbf{Updates:}]
   
\EndNotes
\rule[1em]{\linewidth}{0.4pt}\vspace{-1cm}



\DayNotes{Day 12 | Act 09 | (09/25)} 
    \item[\quad\textbf{Topic:}] 
    
    \item[\quad\textbf{Material:}]
    
    \item[\quad\textbf{Warm Up:}]

    \item[\quad\textbf{Notes:}] 

    \item[\quad\textbf{Updates:}]
   
\EndNotes
\rule[1em]{\linewidth}{0.4pt}\vspace{-1cm}
\DayNotes{Day XX ----------------- (12/11)} 
    \item[\quad\textbf{Topic:}] Last Day of the semester!
\EndNotes
\rule[1em]{\linewidth}{0.4pt}\vspace{-1cm}

\end{document}