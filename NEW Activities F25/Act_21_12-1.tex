\documentclass[12pt]{article}                         
\pagestyle{plain}

\usepackage{amsmath}     % Enhanced math environments (e.g., align).
\usepackage{amsfonts}    % Math fonts (e.g., \mathfrak{}).
\usepackage{amstext}     % Text inside math mode (e.g., \text{where}).
\usepackage{amssymb}     % Extra math symbols (e.g., \mathbb{R}).
\usepackage{array}       % Advanced table/array column definitions.
\usepackage{circledtext} % Puts text inside a circle (e.g., \circledtext{A}).
\usepackage{comment}     % Include/exclude blocks of text.
\usepackage{enumerate}   % Customize itemized/numbered lists.
\usepackage{graphicx}    % Include images/graphics (\includegraphics).
\usepackage{latexsym}    % Access to basic LaTeX symbols.
\usepackage{multicol}    % Allows text columns on a page.
\usepackage{pgfplots}    % Create scientific plots from data (based on TikZ).
\usepackage{tabularx}    % Tables that stretch to page width.
\usepackage{tasks}       % Create multi-column lists.
\usepackage{textcomp}    % Provides many text symbols (e.g., \textcelsius).
\usepackage{tikz}        % Create vector graphics and diagrams.
\usepackage{xcolor}      % Define and use colors.
\usepackage{fancyhdr}
\usepackage{tcolorbox}
\usepackage{enumitem}
\pgfplotsset{compat=1.18}

\usepackage[
  letterpaper,
  left=0.8in,
  right=0.8in,
  textheight=9.5in,
  bmargin=0.5in  % Adjust this value to push the footer down
]{geometry}
\pagestyle{fancy}
\fancyhf{} % Clear all header and footer fields
\fancyhead[L]{Your Name:} % Left header with name
\fancyhead[R]{December 2nd 2025} % Right header with date
\renewcommand{\headrulewidth}{0.4pt} % Horizontal line below the header

\begin{document}

% Main title
\begin{center}
    \Large \textbf{Math 115E Activity 21} \\
    \vspace{0.2cm}
    \normalsize Chapter 7: Polynomials \\
    \normalsize Multiplicities
\end{center}
\vspace{-0.5cm}
\subsection*{Multiplicities of Polynomials}

\begin{tcolorbox}[
    width=\linewidth,
    colframe=black,         % Border color
    colback=white,          % Background color
    boxrule=0.5pt,          % Border thickness
    left=1mm, right=1.1mm,  % Horizontal padding
    top=1mm, bottom=1mm,    % Vertical padding
    arc=2mm                 % Corner radius
]
\textbf{Reminder:} 
\textit{Given a function $f(x)$, 
\begin{itemize} 
    \item The x-intercepts are the coordinate points of the form $(x,0)$ on the graph $f(x)$ 
    \item The y-intercept is the coordinate points of the form $(0,f(0))$ on the graph of $f(x)$
\end{itemize}}
\end{tcolorbox}
\begin{tcolorbox}[
    width=\linewidth,
    colframe=black,         % Border color
    colback=white,          % Background color
    boxrule=0.5pt,          % Border thickness
    left=1mm, right=1.1mm,    % Horizontal padding
    top=1mm, bottom=1mm,    % Vertical padding
    arc=2mm                 % Corner radius
]
\textbf{Definition:} 
\textit{Given a polynomial $f(x)$, the number of times a given term $(x-c)$ 
appears in the factored form of $f(x)$ is called the \textbf{multiplicity}}
\end{tcolorbox}
\noindent\\
Example: If we have the polynomial: $g(x) = (x-1)(x-2)^4(x+3)^3(x+4)^2$
\begin{itemize}
    \item Then we can say the following solutions are $x= 1, x= 2,x= -3,x= -4$ 
    \item Now, notice that: $x=\hphantom{-}1$ has a multiplicity of 1, and $x=\hphantom{-}2$ has a multiplity of 4 \\
    \hspace*{90pt} $x=-3$ has a multiplicity of 3, and $x=-4$ has a multiplity of 2
    \item The y-intercepts is at $f(0)=(0-1)(0-2)^4(0+3)^3(0+4)^2=(-1)(-2)^4(3)^3(4)^2=-6912$
    \item The x-intercepts are at $0=f(x)$ which are $(1,0), (2,0),(-3,0),(-4,0)$
\end{itemize}
\noindent\\
For the following problems, find the x-intercepts and their multiplicities, and the y-intercepts
\begin{enumerate}
    \item[\#1] $f(x) = (x-2)^3(x-1)$
    \\\\\\
    \item[\#2] $f(x) = (x+1)^2(x-1)$
    \\\\\\
    \item[\#3] $f(x) = (x^2-4)(x+3)^3$
    \\\\\\
    \item[\#4] $f(x) = (x-1)^4(x^2+3)(x+2)^2$
    \\\\\\
    \item[\#5] $f(x) = x(x^2-2)(x-3)^3(x+4)^2$
\end{enumerate}


\end{document}