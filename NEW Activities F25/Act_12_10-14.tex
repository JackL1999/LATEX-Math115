\documentclass[12pt]{article}                         
\pagestyle{plain}

\usepackage{amsmath}     % Enhanced math environments (e.g., align).
\usepackage{amsfonts}    % Math fonts (e.g., \mathfrak{}).
\usepackage{amstext}     % Text inside math mode (e.g., \text{where}).
\usepackage{amssymb}     % Extra math symbols (e.g., \mathbb{R}).
\usepackage{array}       % Advanced table/array column definitions.
\usepackage{circledtext} % Puts text inside a circle (e.g., \circledtext{A}).
\usepackage{comment}     % Include/exclude blocks of text.
\usepackage{enumerate}   % Customize itemized/numbered lists.
\usepackage{geometry}    % Adjusts page margins and layout.
\usepackage{graphicx}    % Include images/graphics (\includegraphics).
\usepackage{latexsym}    % Access to basic LaTeX symbols.
\usepackage{multicol}    % Allows text columns on a page.
\usepackage{pgfplots}    % Create scientific plots from data (based on TikZ).
\usepackage{tabularx}    % Tables that stretch to page width.
\usepackage{tasks}       % Create multi-column lists.
\usepackage{textcomp}    % Provides many text symbols (e.g., \textcelsius).
\usepackage{tikz}        % Create vector graphics and diagrams.
\usepackage{xcolor}      % Define and use colors.
\usepackage{fancyhdr}
\usepackage{tcolorbox}
\geometry{a4paper, margin=1in}
\pagestyle{fancy}
\fancyhf{} % Clear all header and footer fields
\fancyhead[L]{Your Name} % Left header with name
\fancyhead[R]{October 14th 2025} % Right header with date
\renewcommand{\headrulewidth}{0.4pt} % Horizontal line below the header

\begin{document}

% Main title
\begin{center}
    \Large \textbf{Math 115E Activity 12} \\
    \vspace{0.2cm}
    \normalsize Chapter 4 Section 5 \\
    \normalsize Determining the Linear Function
\end{center}

\begin{tcolorbox}[
    width=\linewidth,
    colframe=black,         % Border color
    colback=white,          % Background color
    boxrule=0.5pt,          % Border thickness
    left=1mm, right=1.1mm,    % Horizontal padding
    top=1mm, bottom=1mm,    % Vertical padding
    arc=2mm                 % Corner radius
]
\textbf{Definition.} 
\textit{A linear function can be expressed by $f(x)=mx+b$,
where $x$ is the input, $m$ is the slope, and $b$ is the y intercept of $f(x)$.
This is known as slope-intercept form}
\end{tcolorbox}

\vspace{0.2cm}

\begin{tcolorbox}[
    width=\linewidth,
    colframe=black,         % Border color
    colback=white,          % Background color
    boxrule=0.5pt,          % Border thickness
    left=1mm, right=1.1mm,    % Horizontal padding
    top=1mm, bottom=1mm,    % Vertical padding
    arc=2mm                 % Corner radius
]
\textbf{Helpful steps.} 
\textit{There are two slightly different methods to find the linear function
\begin{itemize}
    \item Use point-slope form $(y-y_0)=m(x-x_0)$: Plug in a point $(x_0,y_0)$ and $m$
    \item Use slope-intercept form $y=mx+b$: Plug in a point $(x,y)$ then solve for b
\end{itemize}
}
\end{tcolorbox}
\newcolumntype{W}{>{\centering\arraybackslash}p{0.7cm}} 
\section*{Section 1: Find the linear function}
\begin{minipage}[t]{0.4\textwidth}
    \begin{enumerate}[\#1]
        \item $(-1,1)$ and $(3,6)$
        \\\\\\\\\\\\\\\\\\\\
        \item $(-1,1)$ and $(3,1)$
        \\\\\\\\\\\\\\\\\\\\
        \item $(0,0)$ and $(2,-2)$

    \end{enumerate}
\end{minipage}
\begin{minipage}[t]{0.4\textwidth}
    \begin{enumerate}[\#1]
        \setcounter{enumi}{3} % continues numbering
        \item $(4,5)$ and $(-10,-4)$
        \\\\\\\\\\\\\\\\\\\\
        \item $(0,12)$ and $(7,0)$
        \\\\\\\\\\\\\\\\\\\\
        \item $(-4,-5)$ and $(6,2)$

     \end{enumerate}
\end{minipage}
\begin{minipage}[t]{0.4\textwidth}
    \begin{enumerate}[\#1]
        \setcounter{enumi}{6} % continues numbering
        \item $(2,10)$ and $(-4,-8)$
        \\\\\\\\\\\\\\\\\\\\
        \item $(5,5)$ and $(5,-1)$
        \\\\\\\\\\\\\\\\\\\\
        \item $(-20,30)$ and $(-40,90)$
         \end{enumerate}
\hfill
\end{minipage}

\vspace{10cm}

\begin{center}
    \Large \textbf{Math 115E Activity 12} \\
    \vspace{0.2cm}
    \normalsize Chapter 4 Section 5 \\
    \normalsize Determining the Linear Function
\end{center}

\begin{tcolorbox}[
    width=\linewidth,
    colframe=black,         % Border color
    colback=white,          % Background color
    boxrule=0.5pt,          % Border thickness
    left=1mm, right=1.1mm,    % Horizontal padding
    top=1mm, bottom=1mm,    % Vertical padding
    arc=2mm                 % Corner radius
]
\textbf{Definition.} 
\textit{An inequality looks just like an equation, except that in place of
the equal sign, we have one of the symbols $<,>,\leq, \geq$. Giving us a range of values not just one\\
Example: $4x-3=1 \rightarrow x=1$ compared to $4x-3\geq1 \rightarrow x\geq1$}

\end{tcolorbox}

\vspace{0.2cm}

\begin{tcolorbox}[
    width=\linewidth,
    colframe=black,         % Border color
    colback=white,          % Background color
    boxrule=0.5pt,          % Border thickness
    left=1mm, right=1.1mm,    % Horizontal padding
    top=1mm, bottom=1mm,    % Vertical padding
    arc=2mm                 % Corner radius
]
\textbf{Helpful steps.} 
\textit{There are two slightly different methods to find the linear function
\begin{itemize}
    \item To solve: Isolate $x$ terms to one side, then simplify both sides
    \item If you divide both sides by a negitive number, then switch the inequality signs\\
    Example: We start with $leq$ and becomes $\geq$ and then $>$ becomes $<$ and vise versa
\end{itemize}
}
\end{tcolorbox}
\newcolumntype{W}{>{\centering\arraybackslash}p{0.7cm}} 
\section*{Section 2: Solve the Expressions}
\begin{minipage}[t]{0.35\textwidth}
    \begin{enumerate}[\#1]
        \item $2x + 3 = 5$
        \\\\\\\\\\\\\\
        \item $4x - 3 \geq 9$
        \\\\\\\\\\\\\\
        \item $5x - 1 < x + 8$
        \\\\\\\\\\\\\\
        \item $\frac{5}{6}x-12 = 4$
    \end{enumerate}
\end{minipage}
\begin{minipage}[t]{0.4\textwidth}
    \begin{enumerate}[\#1]
        \setcounter{enumi}{4} % continues numbering
        \item $4(x-5) + 2 \geq 2 + 2x$
        \\\\\\\\\\\\\\
        \item $4 - \frac{4}{5}x = 4$
        \\\\\\\\\\\\\\
        \item $\frac{3}{2}x-6 \leq \frac{4}{5} + 8$
        \\\\\\\\\\\\\\
        \item $-(-x-4) = \frac{1}{2}(x-2)$
     \end{enumerate}
\end{minipage}
\begin{minipage}[t]{0.3\textwidth}
    \begin{enumerate}[\#1]
        \setcounter{enumi}{8} % continues numbering
        \item $\frac{4}{3}x-\frac{1}{5}=\frac{5}{6}x$
        \\\\\\\\\\\\\\
        \item $5x+1 = 5x - x + 1$
        \\\\\\\\\\\\\\
        \item $6x - 4 < 6x - 5$
        \\\\\\\\\\\\\\
        \item $5x - 2 > 5x - 2$
        \end{enumerate}
\hfill
\end{minipage}

\end{document}