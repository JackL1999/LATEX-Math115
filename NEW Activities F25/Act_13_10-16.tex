\documentclass[12pt]{article}                         
\pagestyle{plain}

\usepackage{amsmath}     % Enhanced math environments (e.g., align).
\usepackage{amsfonts}    % Math fonts (e.g., \mathfrak{}).
\usepackage{amstext}     % Text inside math mode (e.g., \text{where}).
\usepackage{amssymb}     % Extra math symbols (e.g., \mathbb{R}).
\usepackage{array}       % Advanced table/array column definitions.
\usepackage{circledtext} % Puts text inside a circle (e.g., \circledtext{A}).
\usepackage{comment}     % Include/exclude blocks of text.
\usepackage{enumerate}   % Customize itemized/numbered lists.
\usepackage{geometry}    % Adjusts page margins and layout.
\usepackage{graphicx}    % Include images/graphics (\includegraphics).
\usepackage{latexsym}    % Access to basic LaTeX symbols.
\usepackage{multicol}    % Allows text columns on a page.
\usepackage{pgfplots}    % Create scientific plots from data (based on TikZ).
\usepackage{tabularx}    % Tables that stretch to page width.
\usepackage{tasks}       % Create multi-column lists.
\usepackage{textcomp}    % Provides many text symbols (e.g., \textcelsius).
\usepackage{tikz}        % Create vector graphics and diagrams.
\usepackage{xcolor}      % Define and use colors.
\usepackage{fancyhdr}
\usepackage{tcolorbox}
\geometry{a4paper, margin=1in}
\pagestyle{fancy}
\fancyhf{} % Clear all header and footer fields
\fancyhead[L]{Your Name} % Left header with name
\fancyhead[R]{October 16th 2025} % Right header with date
\renewcommand{\headrulewidth}{0.4pt} % Horizontal line below the header

\begin{document}

% Main title
\begin{center}
    \Large \textbf{Math 115E Activity 12} \\
    \vspace{0.2cm}
    \normalsize Chapter 5 Prereqs \\
    \normalsize Number puzzles
\end{center}

\begin{tcolorbox}[
    width=\linewidth,
    colframe=black,         % Border color
    colback=white,          % Background color
    boxrule=0.5pt,          % Border thickness
    left=1mm, right=1.1mm,    % Horizontal padding
    top=1mm, bottom=1mm,    % Vertical padding
    arc=2mm                 % Corner radius
]
\textbf{Helpful tips.} 
\textit{All of the solutions to the following problems will be integers\\
meaning numbers like ...-3,-2,-1,0,1,2,3..., so no fractions and no decimals at all}
\end{tcolorbox}

\section*{Section 1: Finding two numbers to solve the puzzle}
    \begin{enumerate}[\#1]
        \item Give me two numbers such that: 
        they both multiply to -6 and yet both add to 1
        \\\\\\\\\\
        \item Give me two numbers such that: 
        they both multiply to -6 and yet both add to -1
        \\\\\\\\\\
        \item Give me two numbers such that: 
        they both multiply to 20 and yet both add to -9
        \\\\\\\\\\
        \item Give me two numbers such that: 
        they both multiply to 6 and yet both add to 7
        \\\\\\\\\\
        \item Give me two numbers such that: 
        they both multiply to -80 and yet both add to -2
        \\\\\\\\\\
        \item Give me two numbers such that: 
        they both multiply to 42 and yet both add to 13

    \end{enumerate}
\vspace{4cm}
\section*{Section 2: Finding two numbers again but harder}
    \begin{enumerate}[\#1]
        \item Find two numbers such that: \\
        the first number and the second number both multiply to -15,\\
        and the first plus the product of 2 and the second number give us -7
        \\\\\\\\\\
        \item Find two numbers such that: \\
        the first number and the second number both multiply to 24,\\
        and the first plus the product of 3 and the second number give us 22
        \\\\\\\\\\
        \item Find two numbers such that: \\
        the first number and the second number both multiply to 4,\\
        and the first plus the product of 2 and the second number give us 6
        \\\\\\\\\\
        \item Find two numbers such that: \\
        the first number and the second number both multiply to 6,\\
        and the first plus the product of 4 and the second number give us -14
        \\\\\\\\\\
        \item Find two numbers such that: \\
        the first number and the second number both multiply to -2,\\
        and the first plus the product of 2 and the second number give us 3
        \\\\\\\\\\
        \item Find two numbers such that: \\
        the first number and the second number both multiply to -16,\\
        and the first plus the product of 3 and the second number give us -8
    \end{enumerate}
\end{document}