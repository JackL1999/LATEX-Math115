\documentclass{article}
\usepackage{amsmath}
\usepackage{amssymb}
\usepackage{fancyhdr}
\usepackage[left=1in, right=1in, top=1.5in, bottom=1in]{geometry}% Set up custom header
\pagestyle{fancy}
\fancyhf{} % Clear all header and footer fields
\fancyhead[L]{Your Name} % Left header with name
\fancyhead[R]{August 26th 2025} % Right header with date
\renewcommand{\headrulewidth}{0.4pt} % Horizontal line below the header

\begin{document}

% Main title
\begin{center}
    \Large \textbf{Math 115E Activity 1} \\
    \vspace{0.2cm}
    \normalsize Chapter 1 Section 1 \\
    \normalsize Number Systems and Solution Sets
\end{center}
\vspace{1cm} % Add space between the title and the first exercise

% Exercise 1
\section*{What are the different types of numbers?}
\begin{itemize}
    \item \textbf{Natural Numbers}: $1, 2, 3, 4, \ldots$ We use the
symbol $\mathbb{N}$ to refer to the natural numbers.
    
    \item \textbf{Integers}: $\ldots, -4, -3, -2, -1, 0, 1, 2, 3, 4, \ldots$ We use the symbol $\mathbb{Z}$ to refer to the integers.

    \item \textbf{Rational Numbers}: A number that can be expressed as a fraction $p/q$ of two integers, \\
    where $p$ is the numerator and $q$ is the non-zero denominator. \\
    We use the symbol Q for the rational numbers.
    
    \item \textbf{Irrational Numbers}: A number that cannot be expressed as a simple fraction. \\
    Its decimal representation is non-terminating and non-repeating. \\
    Examples include $\pi$, $e$, and $\sqrt{2}$.We don’t actually have a
    fancy symbol to describe the irrationals.

\end{itemize}

\vspace{0.5cm} % Add space between exercises

% Exercise 2
\section*{Practice Problems}
In this section you will give examples of each type of number, without clear repeats
\begin{itemize}
    \item \textbf{5 Natural Numbers}: 
    \vspace{0.5cm}
    \item \textbf{5 Integers}: 
    \vspace{0.5cm}
    \item \textbf{5 Rational Numbers}: 
    \vspace{0.5cm}
    \item \textbf{2 Irrational Numbers}: 
    \vspace{0.5cm}
\end{itemize}

\end{document}