\documentclass[12pt]{article}                         
\pagestyle{plain}

\usepackage{amsmath}     % Enhanced math environments (e.g., align).
\usepackage{amsfonts}    % Math fonts (e.g., \mathfrak{}).
\usepackage{amstext}     % Text inside math mode (e.g., \text{where}).
\usepackage{amssymb}     % Extra math symbols (e.g., \mathbb{R}).
\usepackage{array}       % Advanced table/array column definitions.
\usepackage{circledtext} % Puts text inside a circle (e.g., \circledtext{A}).
\usepackage{comment}     % Include/exclude blocks of text.
\usepackage{enumerate}   % Customize itemized/numbered lists.
\usepackage{geometry}    % Adjusts page margins and layout.
\usepackage{graphicx}    % Include images/graphics (\includegraphics).
\usepackage{latexsym}    % Access to basic LaTeX symbols.
\usepackage{multicol}    % Allows text columns on a page.
\usepackage{pgfplots}    % Create scientific plots from data (based on TikZ).
\usepackage{tabularx}    % Tables that stretch to page width.
\usepackage{tasks}       % Create multi-column lists.
\usepackage{textcomp}    % Provides many text symbols (e.g., \textcelsius).
\usepackage{tikz}        % Create vector graphics and diagrams.
\usepackage{xcolor}      % Define and use colors.
\usepackage{fancyhdr}
\usepackage{tcolorbox}
\geometry{a4paper, margin=1in}
\pagestyle{fancy}
\fancyhf{} % Clear all header and footer fields
\fancyhead[L]{Your Name} % Left header with name
\fancyhead[R]{October 23th 2025} % Right header with date
\renewcommand{\headrulewidth}{0.4pt} % Horizontal line below the header

\begin{document}

% Main title
\begin{center}
    \Large \textbf{Math 115E Activity 14} \\
    \vspace{0.2cm}
    \normalsize Chapter 5 Section 1-2 \\
    \normalsize Factoring Quadratics Part 2
\end{center}

\noindent
\section*{How to factor quadratic equations}
\noindent
\begin{minipage}[t]{0.48\textwidth}
\begin{tcolorbox}[
    width=\linewidth,
    colframe=black,         % Border color
    colback=white,          % Background color
    boxrule=0.5pt,          % Border thickness
    left=1mm, right=1.1mm,    % Horizontal padding
    top=1mm, bottom=1mm,    % Vertical padding
    arc=2mm                 % Corner radius
]
\textbf{Quadratic factoring when} $\mathbf{a = 1}$: \\ 
\textit{When factoring, the form $x^2 + bx + c$ \\
can be factored as $(x+m)(x+n)$\\
Start with real numbers $m$ and $n$ so: \\
they both multiply to $c$ and both add to $b$\\
There is not a value in front of either $x$}
\end{tcolorbox}
\end{minipage}%
\hfill
\begin{minipage}[t]{0.48\textwidth}
\begin{tcolorbox}[
    width=\linewidth,
    colframe=black,         % Border color
    colback=white,          % Background color
    boxrule=0.5pt,          % Border thickness
    left=1mm, right=1.1mm,    % Horizontal padding
    top=1mm, bottom=1mm,    % Vertical padding
    arc=2mm                 % Corner radius
]
\textbf{Quadratic factoring when} $\mathbf{a \neq 1}$: \\ 
\textit{When factoring, the form $\boldsymbol{a}x^2 + bx + c$ \\
can be factored as $(px+m)(qx+n)$\\
Start with two real numbers such that: \\
multiply to $\boldsymbol{a} \cdot c$ and yet add to $b$ \\
then we re-group the terms and factor}
\end{tcolorbox}
\end{minipage}
\noindent
    Example:Factor $3x^2-11x+10$
    \begin{enumerate}
    \renewcommand{\labelenumi}{}
    \item Step 1: Find the factors of $3*10=30$ that add up to $-11$, (Write the factors if needed)
    \item Step 2: The factors of $30$ are: $\pm(1,30), \pm(2,15),\pm(3,10),\pm(5,6)$, \\
          \hspace*{1.1cm} so the pair that adds to $-11$ is $-5$ and $-6$
    \item Step 3: Rewrite the quadratic:                          \hspace*{120pt} $3x^2+(-6x-5x)+10=0$
    \item Step 4: Regroup so that each group has a common factor: $(3x^2-6x)+(-5x+10)=0$
    \item Step 5: Factor out a common term:                       \hspace*{100pt} $(3x)(x-2)-5(x-2)=0$
    \item Step 6: Factor again with the $x-2$ term:               \hspace*{115pt}  $(3x-5)(x-2)=0 $
    \item DONE: So starting with $3x^2-11x+10=0$, we got $(3x-5)(x-2)=0$
    \item Note: If we arn't able to factor again in Step 5, then go back to Step 3 and swap the factor pair.

    \end{enumerate}
\section*{Factor the following quadratic equations}
\begin{minipage}[t]{0.45\textwidth}
    \begin{enumerate}[\#1)]
        \setcounter{enumi}{0} % continues numbering
        \item  $x^2+{\color{red} 9}x+14$
        \vspace{1em}
        \item  $x^2-8x+7$
        \vspace{1em}
        \item  $x^2+x-{\color{red} 30}$
        \vspace{1em}
        \item  $3x^2+10x+8$
        \vspace{1em}
        \item  $2x^2-9x+10$
        \vspace{1em}
        \item  $2x^2-6x-20$
        \end{enumerate}
\end{minipage}%
\hspace{1cm}
\begin{minipage}[t]{0.45\textwidth}
    \begin{enumerate}[\#1)]
        \setcounter{enumi}{6} % continues numbering
        \item  $9x^2-27x+18$
        \vspace{1em}
        \item  $4x^2-13x+10$
        \vspace{1em}
        \item  $2x^2-13x-7$
        \vspace{1em}
        \item  $4x^2+20x+25$
        \vspace{1em}
        \item  $3x^2-19x+20$
        \vspace{1em}
        \item  $8x^2-6x-9$

    \end{enumerate}
\end{minipage}

\begin{comment}
Notes:
- make ++ to ++, -- to +- table etc
- double check forms
- connect to zeros
- similar thing but a is not 1 as before
- show how graph is affected
- (x-m)(x-n) to (x+m)(x+n)
- backwards and expand to quadratic
- factor vs quadratic formula worksheet
\end{comment}


Find two numbers such that: \\
the first number and the second number both multiply to 6,\\
and the first plus the product of 4 and the second number give us -14
\\\\
\noindent
Expand $(2x+3)(3x-1)$

\end{document}