\documentclass{article}                         
\pagestyle{plain}

\usepackage{amsmath}     % Enhanced math environments (e.g., align).
\usepackage{amsfonts}    % Math fonts (e.g., \mathfrak{}).
\usepackage{amstext}     % Text inside math mode (e.g., \text{where}).
\usepackage{amssymb}     % Extra math symbols (e.g., \mathbb{R}).
\usepackage{array}       % Advanced table/array column definitions.
\usepackage{circledtext} % Puts text inside a circle (e.g., \circledtext{A}).
\usepackage{comment}     % Include/exclude blocks of text.
\usepackage{enumerate}   % Customize itemized/numbered lists.
\usepackage{geometry}    % Adjusts page margins and layout.
\usepackage{graphicx}    % Include images/graphics (\includegraphics).
\usepackage{latexsym}    % Access to basic LaTeX symbols.
\usepackage{multicol}    % Allows text columns on a page.
\usepackage{pgfplots}    % Create scientific plots from data (based on TikZ).
\usepackage{tabularx}    % Tables that stretch to page width.
\usepackage{tasks}       % Create multi-column lists.
\usepackage{textcomp}    % Provides many text symbols (e.g., \textcelsius).
\usepackage{tikz}        % Create vector graphics and diagrams.
\usepackage{xcolor}      % Define and use colors.
\usepackage{fancyhdr}
\usepackage{tcolorbox}
\geometry{a4paper, margin=1in}
\pagestyle{fancy}
\fancyhf{} % Clear all header and footer fields
\fancyhead[L]{Your Name} % Left header with name
\fancyhead[R]{October 23th 2025} % Right header with date
\renewcommand{\headrulewidth}{0.4pt} % Horizontal line below the header

\begin{document}

% Main title
\begin{center}
    \Large \textbf{Math 115E Activity 14} \\
    \vspace{0.2cm}
    \normalsize Chapter 5 Section 1-2 \\
    \normalsize Quadratics
\end{center}

\noindent
\textbf{Section 2: Factoring a quadratic expression}\\
\noindent
\begin{minipage}[t]{0.48\textwidth}
\begin{tcolorbox}[
    width=\linewidth,
    colframe=black,         % Border color
    colback=white,          % Background color
    boxrule=0.5pt,          % Border thickness
    left=1mm, right=1.1mm,    % Horizontal padding
    top=1mm, bottom=1mm,    % Vertical padding
    arc=2mm                 % Corner radius
]
\textbf{Quadratic factoring when $a = 1$: \\} 
\textit{When factoring, the form $x^2 + bx + c$ \\
can be factored as $(x+m)(x+n)$\\
with real numbers $p$ and $q$ such that: \\
they both multiply to $c$ and yet both add to $b$}
\end{tcolorbox}
\end{minipage}%
\hfill
\begin{minipage}[t]{0.48\textwidth}
\begin{tcolorbox}[
    width=\linewidth,
    colframe=black,         % Border color
    colback=white,          % Background color
    boxrule=0.5pt,          % Border thickness
    left=1mm, right=1.1mm,    % Horizontal padding
    top=1mm, bottom=1mm,    % Vertical padding
    arc=2mm                 % Corner radius
]
\textbf{Quadratic factoring when $a \neq 1$: \\} 
\textit{When factoring, the form $\boldsymbol{a}x^2 + bx + c$ \\
can be factored as $(x+m)(x+n)$\\
with real numbers $p$ and $q$ such that: \\
they both multiply to $\boldsymbol{a} \cdot c$ and yet both add to $b$}
\end{tcolorbox}
\end{minipage}


\begin{minipage}[t]{0.45\textwidth}
    \begin{enumerate}
        \item Factor $x^2 + x - 6$
        \\\\\\\\\\\\\\
        \item Factor $x^2 - x - 6$
        \\\\\\\\\\\\\\
        \item Factor $x^2 - 9x + 20$
        \\\\\\\\\\\\\\
        \item Factor $x^2 + 7x + 6$
        \\\\\\\\\\\\\\
        \item Factor $x^2 - 2x - 80$
        \\\\\\\\\\\\\\
        \item Factor $x^2 - 13x + 42$
        
        
    \end{enumerate}
\end{minipage}%
\hfill
\begin{minipage}[t]{0.45\textwidth}
    \begin{enumerate}
        \setcounter{enumi}{4} % continues numbering
        \item Factor $x^2 - 4$
        \\\\\\\\\\\\\\
        \item Factor $x^2 + 2x - 120$
        \\\\\\\\\\\\\\
        \item Factor $x^2 + 22x + 120$
        \\\\\\\\\\\\\\
        \item Factor $x^2 + 18x + 32$
        \\\\\\\\\\\\\\
        \item Factor $x^2 + 26x + 160$
        \\\\\\\\\\\\\\
        \item Factor $x^2 + 33x + 200$
        

    \end{enumerate}
\end{minipage}

\begin{comment}
Notes:
- make ++ to ++, -- to +- table etc
- double check forms
- connect to zeros
- similar thing but a is not 1 as before
- show how graph is affected
- (x-m)(x-n) to (x+m)(x+n)
- backwards and expand to quadratic
- factor vs quadratic formula worksheet
\end{comment}
\end{document}