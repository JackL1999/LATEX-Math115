\documentclass{article}
\usepackage{amsmath}
\usepackage{amssymb}
\usepackage{fancyhdr}
\usepackage[utf8]{inputenc}
\usepackage{tcolorbox}
\usepackage[left=1in, right=1in, top=1.5in, bottom=1in]{geometry}
\usepackage{tikz}
\usepackage{enumerate}
\usepackage{enumitem}
\pagestyle{fancy}
\fancyhf{} % Clear all header and footer fields
\fancyhead[L]{Your Name} % Left header with name
\fancyhead[R]{September 11th 2025} % Right header with date
\renewcommand{\headrulewidth}{0.4pt} % Horizontal line below the header
\newmdtheoremenv{theo}{Theorem}

\begin{document}

% Main title
\begin{center}
    \Large \textbf{Math 115E Activity 5} \\
    \vspace{0.2cm}
    \normalsize Chapter 3 Section 2 \\
    \normalsize Number Systems and Solution Sets
\end{center}
\vspace{1cm} % Add space between the title and the first exercise

\begin{tcolorbox}[
    width=\linewidth,
    colframe=black,         % Border color
    colback=white,          % Background color
    boxrule=0.5pt,          % Border thickness
    left=1mm, right=1mm,    % Horizontal padding
    top=1mm, bottom=1mm,    % Vertical padding
    arc=2mm                 % Corner radius
]
\textbf{Definition.} \textit{Given a graph in the coordinate plane, coordinate points of the form (x,0) on the curve are x-intercepts, and coordinate points of the form (0,y) on the curve are y-intercepts.}
\end{tcolorbox}

\begin{theo}
Given a graph in the coordinate plane, 
coordinate points of the form (x,0) on the curve are x-intercepts,
 and coordinate points of the form (0,y) on the curve are y-intercepts.
\end{theo}

\end{document}