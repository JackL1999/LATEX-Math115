\documentclass{article}
\usepackage{amsmath}
\usepackage{amssymb}
\usepackage{fancyhdr}
\usepackage[utf8]{inputenc}
\usepackage{tcolorbox}
\usepackage[left=1in, right=1in, top=1.5in, bottom=1in]{geometry}
\usepackage{tikz}
\usepackage{enumerate}
\usepackage{enumitem}
\usepackage{mdframed}
\usepackage{lipsum}
\usepackage{fancybox}

\newmdtheoremenv{theo}{Theorem}
\newtheorem{definition}{Definition}
\surroundwithmdframed{definition}

\begin{document}

\begin{theo}
Given a graph in the coordinate plane, 
coordinate points of the form (x,0) on the curve are x-intercepts,
 and coordinate points of the form (0,y) on the curve are y-intercepts.
\end{theo}

\begin{tcolorbox}[
    width=\linewidth,
    colframe=black,         % Border color
    colback=white,          % Background color
    boxrule=0.5pt,          % Border thickness
    left=1mm, right=1mm,    % Horizontal padding
    top=1mm, bottom=1mm,    % Vertical padding
    arc=2mm                 % Corner radius
]
\textbf{Definition.} \textit{Given a graph in the coordinate plane, coordinate points of the form (x,0) on the curve are x-intercepts, and coordinate points of the form (0,y) on the curve are y-intercepts.}
\end{tcolorbox}

\begin{tcolorbox}
\textbf{Definition.} \textit{Given a graph in the coordinate plane, coordinate points of the form (x,0) on the curve are x-intercepts, and coordinate points of the form (0,y) on the curve are y-intercepts.}
\end{tcolorbox}

\section*{Example Definitions}

\begin{definition}
  A **prime number** is a natural number greater than 1 that has no positive divisors other than 1 and itself.
\end{definition}

\begin{definition}
  An **Euclidean space** is the fundamental space of geometry, intended to represent the three-dimensional space of ordinary experience.
\end{definition}

\begin{definition}
  A **group** is a set of elements together with an operation that combines any two of its elements to form a third element, and that satisfies four conditions called the group axioms.
\end{definition}

\section*{Example Definitions}

\fbox{\textbf{Definition}\\[1ex]
A **prime number** is a natural number greater than 1 that has no positive divisors other than 1 and itself.}

\vspace{1em}

\fbox{\textbf{Definition}\\[1ex]
An **Euclidean space** is the fundamental space of geometry, intended to represent the three-dimensional space of ordinary experience.}

\section*{Simple Boxed Definition}

\fbox{\parbox{0.9\linewidth}{
  \textbf{Definition}\\[0.5em]
  A **prime number** is a natural number greater than 1 that has no positive divisors other than 1 and itself.
}}

\vspace{1em}

\fbox{\parbox{0.9\linewidth}{
  \textbf{Definition}\\[0.5em]
  An **Euclidean space** is the fundamental space of geometry, intended to represent the three-dimensional space of ordinary experience.
}}

\begin{mdframed}
\textbf{Definition.} \textit{Given a graph in the coordinate plane, coordinate points of the form (x,0) on the curve are x-intercepts, and coordinate points of the form (0,y) on the curve are y-intercepts.}
\end{mdframed}

\end{document}