\documentclass[12pt]{article}                         
\pagestyle{plain}

\usepackage{amsmath}     % Enhanced math environments (e.g., align).
\usepackage{amsfonts}    % Math fonts (e.g., \mathfrak{}).
\usepackage{amstext}     % Text inside math mode (e.g., \text{where}).
\usepackage{amssymb}     % Extra math symbols (e.g., \mathbb{R}).
\usepackage{array}       % Advanced table/array column definitions.
\usepackage{circledtext} % Puts text inside a circle (e.g., \circledtext{A}).
\usepackage{comment}     % Include/exclude blocks of text.
\usepackage{enumerate}   % Customize itemized/numbered lists.
\usepackage{graphicx}    % Include images/graphics (\includegraphics).
\usepackage{latexsym}    % Access to basic LaTeX symbols.
\usepackage{multicol}    % Allows text columns on a page.
\usepackage{pgfplots}    % Create scientific plots from data (based on TikZ).
\usepackage{tabularx}    % Tables that stretch to page width.
\usepackage{tasks}       % Create multi-column lists.
\usepackage{textcomp}    % Provides many text symbols (e.g., \textcelsius).
\usepackage{tikz}        % Create vector graphics and diagrams.
\usepackage{xcolor}      % Define and use colors.
\usepackage{fancyhdr}
\usepackage{tcolorbox}
\usepackage{enumitem}

\usepackage[
  letterpaper,
  left=0.8in,
  right=0.8in,
  textheight=9.5in,
  bmargin=0.5in  % Adjust this value to push the footer down
]{geometry}
\pagestyle{fancy}
\fancyhf{} % Clear all header and footer fields
\fancyhead[L]{Your Name:} % Left header with name
\fancyhead[R]{Novemeber 18th 2025} % Right header with date
\renewcommand{\headrulewidth}{0.4pt} % Horizontal line below the header

\begin{document}

% Main title
\begin{center}
    \Large \textbf{Math 115E Activity XX} \\
    \vspace{0.2cm}
    \normalsize Old Activity 11-5 and 11-7
\end{center}
\vspace{-0.5cm}
\subsection*{Graphing Quadratic Functions}
\begin{tcolorbox}[
    width=\linewidth,
    colframe=black,         % Border color
    colback=white,          % Background color
    boxrule=0.5pt,          % Border thickness
    left=1mm, right=1.1mm,    % Horizontal padding
    top=1mm, bottom=1mm,    % Vertical padding
    arc=2mm                 % Corner radius
]
\textbf{Definition:} 
\textit{A function $f(x)$ of the form $f(x)=a_nx^n+a_{n-1}x^{n-1}+\cdots+a_2x^2+a_1x+a_0$\\
where $a_n, a_{n-1},\dots,a_2,a_1,a_0$ are real numbers and $n$ is a non-negative integer}
\end{tcolorbox}

\noindent\\
For the following questions, circle which functions are polynomials based on the Definition above
\begin{minipage}[t]{0.30\textwidth}
    \begin{enumerate}
        \item[\#1]  $f(x) = x^2+\pi$
        \vspace{0.75em}
        \item[\#2]  $f(x) = \frac{1}{5}x-x^2$
        \vspace{0.75em}
        \item[\#3]  $f(x) = x^e - 2$
        \vspace{0.75em}
        \item[\#4]  $f(x) = x^{1/3}-x^2$
    \end{enumerate}
\end{minipage}
\hspace{1cm}
\begin{minipage}[t]{0.30\textwidth}
    \begin{enumerate}
        \item[\#5]  $f(x) = \frac{1}{x^2+x}$
        \vspace{0.75em}
        \item[\#6]  $f(x) = -\pi x^2 -ex$
        \vspace{0.75em}
        \item[\#7]  $f(x) = e^2 x - \pi$
        \vspace{0.75em}
        \item[\#8]  $f(x) = x^3 -x^2 + 1$
    \end{enumerate}
\end{minipage}
\hspace{1cm}
\begin{minipage}[t]{0.30\textwidth}
    \begin{enumerate}
        \item[\#9]  $f(x) = 2^x -x^2$
        \vspace{0.75em}
        \item[\#10]  $f(x) = sin(x) -x$
        \vspace{0.75em}
        \item[\#11]  $f(x) = x^{-1}+x^{-2}$
        \vspace{0.75em}
        \item[\#12]  $f(x) = 3x^{\pi}-x-1$
    \end{enumerate}
\end{minipage}\\\\\\
\noindent
For each problem you did NOT circle, briefly explain why it was not a polynomial \\
\\
\vspace{6cm}
\\
For the following, write down new examples of functions that are not already written above
\begin{itemize}
    \item Two functions that are polynomials\\\\\\
    \item Two functions that are NOT polynomials
\end{itemize}
\vspace{2cm}
\begin{tcolorbox}[
    width=\linewidth,
    colframe=black,         % Border color
    colback=white,          % Background color
    boxrule=0.5pt,          % Border thickness
    left=1mm, right=1.1mm,    % Horizontal padding
    top=1mm, bottom=1mm,    % Vertical padding
    arc=2mm                 % Corner radius
]
\textbf{Reminder:} 
\textit{Given a function $f(x)$, coordinate points of the form $(x,0)$ on the graph $f(x)$ are the x-intercepts, and coordinate points of the form $(0,f(0))$ on the graph of $f(x)$ are the y-intercepts}
\end{tcolorbox}
\begin{tcolorbox}[
    width=\linewidth,
    colframe=black,         % Border color
    colback=white,          % Background color
    boxrule=0.5pt,          % Border thickness
    left=1mm, right=1.1mm,    % Horizontal padding
    top=1mm, bottom=1mm,    % Vertical padding
    arc=2mm                 % Corner radius
]
\textbf{Definition:} 
\textit{Given a polynomial $f(x)$, the number of times a given term $(x-c)$ 
appears in the factored form of $f(x)$ is called the \textbf{multiplicity}}
\end{tcolorbox}
\end{document}